\documentclass{beamer}
\usepackage[slovene]{babel}
\title{Napredna računalniška orodja - DN1}
\author{Mihaela Stojanovska}
\date{2024}

\begin{document}

\begin{frame}
    \titlepage
\end{frame}

\begin{frame}{Kazalo}
    \tableofcontents
\end{frame}


\begin{frame}
\section{Naloga1\_1.txt}
\frametitle{Naloga1\_1.txt}
    \begin{itemize}
        \item Prva vrstica vsebuje ime podatkov (časovne vrednosti).
        \item Druga vrstica določa 100 vrstic z 1 podatkovno točko na vrstico.
        \item Vsaka vrstica ima eno samo časovno vrednost, shranjeno v vektorju $t$.
    \end{itemize}
    \vspace{0.5cm}
    \textbf{MATLAB Function Used:} \texttt{importdata} \\
    \begin{itemize}
        \item Sintaksa: importdata(filename, delimiterIn, headerlinesIn);
        \item Vhodi: $filename$ (ime datoteke), $delimiterIn$ (ločilo), $headerlinesIn$ (število vrstic glave).
        \item Output: Izhod: Shrani podatke v $data.data$, iz katerega $t$ vsebuje časovne vrednosti.
    \end{itemize}
\end{frame}


\begin{frame}
\section{Graf $P(t)$}
\frametitle{Graf $P(t)$}
    \includegraphics[width=0.8\textwidth]{nro graf.png}
    \centering
\end{frame}


\begin{frame}
\section{Trapezna metoda}
\frametitle{Trapezna metoda}
    \textbf{Formula za trapezna metoda:}
    \[
    \int_{a}^{b} f(x) \, dx \approx \frac{\Delta x}{2} \left( f(x_0) + 2f(x_1) + 2f(x_2) + \dots + 2f(x_{n-1}) + f(x_n) \right)
    \]
    
    \vspace{0.5cm}
    \textbf{Rezultat} \\
    Rezultat smo dobili: $17.1665$
\end{frame}

\end{document}

